% !TeX spellcheck = en_GB
%!TEX root = ../MainBody.tex

% ???
\chapter{改进与说明}% 
本章主要说明在Legendary模板的基础上(后称为原模板),符合《四川大学硕士、博士学位论文格式(2010版)》\cite{si2010si}(后称为模板2010)要求对应的修改之处。
本章主要说明在此前模板基础上进行的改进与官网模板2010的要求的对应。
\section{本模板的改进}

\subsection{版面修改}
模板2010中描述为:“16开纸打印,正文用小四宋,行距20磅,每行34个汉字。版心145$\times$125mm(此处应该为官网笔误,应为145$\times$215mm。”

原模板为:每行36个字。

\textbf{修改为}:16k正度(国内尺寸):185x260,版心: 145x215mm,对应的上下边距为2.25cm,左右边距为1.95,为了一行34字,调整为1.94。实际上这里的上下边距是页眉页脚到页边的距离,最后上下边分别设为了{1.5cm}和{1.75cm}。

\subsection{字号相关修改}
模板2010中描述为:“一级标题用小三黑体,参考文献用五号宋体”,摘要中“专业用小四号宋体”(样例中“专业”两字置后), “关键词”三个字用小四号黑体

原模板为:一级标题显示效果并非小三(虽然其cls文件设置的是小三),参考文献设置的是小五;摘要中专业为黑体,关键字通过加粗方式定义的

\textbf{修改为}:将一级标题强制设置为小三号大小,同时修改参考文献为五号;摘要中专业改为小四号宋体并将“专业”两字后置,将“关键字”三个字设为小四黑体

\subsection{页码相关修改}
模板2010中描述为:无相关描述。

原模板为:正文之前从封面开始进行罗马数字编号页码(封面不显示页码),目录超过1页时,会产生目录页的问题,并会多一张空白页(加两页目录页就是3页了,打印会使后续出错);目录中显示了目录的页码。

\textbf{修改为}:封面也之后开始进行罗马数字编号,目录页紧接着进行罗马数字编号,去除多余空白页;去除了目录中的目录。

\subsection{参考文献内容样式的修改}
模板2010中描述为:无相关描述。

原模板为:原模板的.bst文件在中文文献多于三个时,使用的是“et al”非“等”

\textbf{修改为}:尝试了多个.bst文件之后,选择了清华大学模板中\cite{jiaogitlatex}的.bst文件\footnote{焦润之 \url{https://gitee.com/jiaorzh/thuthesis}}。英文文献时,作者名首字母大写,姓只保留首字母大写;中文多作者用“等”,英文多作者用“et al”。符合国标。

\subsection{英文摘要处的修改}
模板2010中描述为:无相关描述。

原模板为:有“Presented for MSc Degree \quad Subject: Something and Something”这样的字样

\textbf{修改为}:删掉了“Presented for MSc Degree”,将“Subject”改为了“Major”

\subsection{其他可能因学院细化要求的修改}
模板2010中描述为:所有级别标题“均靠左边顶排”

原模板为:一级标题居中排,一级标题样式为“第x章 ...”;页眉左为“四川大学硕士学位论文”,右为“论文标题”

\textbf{修改为}:一级标题居左排,样式为“x ...”;页眉为“四川大学硕士学位论文”居中(\textbf{注:本处可以根据学院要求和个人喜好进行修改。},如果有同学采用“第x章”的形式有可能会遇到目录中字重叠现象,也是可以搜索解决,或根据main.tex相关注释操作。)